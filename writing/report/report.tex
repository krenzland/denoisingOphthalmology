\documentclass{scrartcl}
\usepackage[utf8]{inputenc}
\usepackage[T1]{fontenc}

\usepackage[american]{babel}
\usepackage[autostyle, english = american]{csquotes}
 \usepackage[%
   backend=biber]{biblatex}
 \addbibresource{../bibliography.bib}

\usepackage[final,tracking=smallcaps,expansion=alltext, protrusion=true]{microtype}
\SetTracking{encoding=*, shape=sc}{50} %latex & friends, page 52
\usepackage{todonotes}
\usepackage{caption}
\usepackage{booktabs}
\usepackage{subcaption}
\usepackage{bm}
\usepackage{amsmath}
\usepackage{amsfonts}
\usepackage{amssymb}
\usepackage{nicefrac}
\usepackage{mathtools} % for \mathclap
\usepackage{hyperref}
\usepackage[noabbrev]{cleveref}
\newcommand{\creflastconjunction}{, and\nobreakspace} % use Oxford comma
%\usepackage{placeins} % for floatbarrier

\usepackage{tikz}
\usetikzlibrary{arrows, positioning, shapes.geometric}
\usetikzlibrary{calc}
\newcommand{\img}{f} % TODO: Format

\begin{document}
\title{Deep-Learning-based Image Denoising in Ophthalmology}
\author{Lukas Krenz}

\maketitle

\section{Introduction}
\todo{Write introduction}
...

\section{Literature Review}

\subsection{Architectures}
\label{sec:architecturs}
Differentiate between methods optimised for quality and speed.
Quality mostly feeds in a upsampled image, filters for in HR-space.
Efficiency uses low-resolution image as input, upscales at end.

We are mostly interested in networks that use a structure that is optimized for efficiency.
All these structures (or at least the cited examples) have one thing in common:
They use the low-resolution image as input.
\begin{description}
\item[Laplacian Pyramid] Structure used by~\cite{LapSRN}.
  Progressive upsampling, predict subband residual for each upsampling step.
  Feature extraction at \textit{coarse}, not \textit{fine} level.
  Same network can do upsampling at different scales.
  Uses deep supervision, each resulting image contributes to the loss.
\item[Hourglass] Used for example by~\cite{Fsrcnn}.
  Structure of form (Feature extraction -> Shrinking -> Mapping -> Expanding -> Deconvolution).
  The filters (except deconvolution) are similar for different upscaling factors, makes retraining easy.
\item[Subpixel Convolution]  Used by~\cite{Espcn}.
  $L$ convolution layers for feature maps, one subpixel convolution layer with $L$ filters.
  In~\cite{Espcn} $L = 3$.
\end{description}

The following architectures/papers are optimised for quality without regarding the run-time:
\begin{description}
\item[Residual Blocks] Used for example by~\cite{EnhanceNet, SRGAN, EDSR}.
   In case of \textbf{EnhanceNet}: structure of form (Conv -> Residual* -> (NNUpsampling -> Conv)* -> Conv) with 10 residual blocks.
  
\item[Deeply Recursive] Used by~\cite{DRCN}.
  Extremely expensive to train (6 days on Titan X).
  Training relies on several tricks.
  Structure of (embedding network -> recursive inference network -> reconstruction network)
\end{description}

More architectures are discussed in the paper~\cite{NTIRE2017} which describes a contest that was won by~\cite{EDSR}.
Most approaches used in this challenge are not real-time capable.

\subsection{Loss functions}
\label{sec:loss}

\begin{description}
\item[MSE] Standard pixelwise mse, optimizes PSNR, leads to blurry images
\item[Charbonnier-Loss] Pixelwise (differentiable approximation of) $L_1$ error, is more robust, used by~\cite{LapSRN}.
  Defined typically as
  \begin{align}
    \label{eq:charbonnier}
    L( \hat{\bm{y}}, \bm{y}; \bm{\theta}) = \sum_n p \left( \bm{\hat{y}} - \bm{y} \right)
    \intertext{with }
    p(\bm{x}) = \sqrt{ \langle x, x \rangle  + \epsilon^2},
  \end{align}
  where $\varepsilon$ is a parameter.
\item[Perceptual Loss] Don't compare pixel-wise but consider difference of filtered images.
  Can be defined for an arbitrary filter map as
  \begin{equation}
    \label{eq:perceptual-loss}
   L_p = \Vert \phi( \bm{\hat{y}} ) - \phi (\bm{y}) \Vert_2^2 .
  \end{equation}
  The filter maps $\phi$ are often taken from the pooling layers of the VGG-19 network, e.g. EnhanceNet uses the second and fifth~\cite{EnhanceNet}.
  Leads to visually better images but decreased PSNR.
  See~\cite{PerceptualLoss} for motivation/details.
\item[Saliency Loss] The same as the perceptual loss but with handcrafted filters.
  One example for this is~\cite{SaliencyGAN} which uses the curvature of the image, distribution of elements in a neighborhood and uniqueness of feature maps.
  This loss is handcrafted for vessel-detection.
\item[Texture Loss] Use patch-wise correlations of perceptive loss, similar to style transfer. See~\cite{EnhanceNet}.
\item[Adversarial Loss] Optimize linear combination of other loss and loss function learned by discriminator/critic network.
  Leads to visually pleasing images, reduced PSNR and might lead to ``imagined'' artifacts.
  First seen in~\cite{SRGAN}.
  Mostly classical GAN architectures are used, might make sense to explore more recent variants for easier training (e.g.\ WGAN-GP).
  
\end{description}
\subsection{Evaluation}\label{sec:evaluation}
\begin{description}
\item[PSNR] Peak-Signal-To-Noise-Ratio. Scaled version of \textsc{mse}, higher values correspond to lower error, defined by:
  \begin{equation}
  \label{eq:psnr}
  \mathrm{PSNR} = -10 \log_{10} (\mathrm{MSE}).
\end{equation}
\item[SSIM] Structural similarity index. Tries to measure perceived quality of images.
\end{description}

Either use metrics such as PSNR/SSIM, compare visually, or use efficiency as pre-processing, e.g. improved segmentation results

\subsection{Relevant results}
Difficult to compare, visual quality does not correspond directly to improved error!
For \textsc{psnr}-results see~\cref{tab:results}.

\begin{table}[]
\centering
\caption{Some Results, all measured in \textsc{psnr}}

\label{tab:results}
\begin{tabular}{@{}llll@{}}
\toprule
Model          & SET5  & SET14 & BSDS100 \\ \midrule
EDSR+~\cite{EDSR} & 32.62 & 28.94 & 27.97 \\
EnhanceNet-E~\cite{EnhanceNet}   & 31.74 & 28.42 & 27.50   \\
LAPSRN~\cite{LapSRN}         & 31.53 & 28.19 & 27.32   \\
DRCN~\cite{DRCN}           & 31.53 & 28.04 & 27.23   \\
ESPCN~\cite{Espcn}       & 30.9  & 27.73 &         \\
FSRCNN~\cite{Fsrcnn}         & 30.71 & 27.7  & 26.97   \\
EnhanceNet-PAT~\cite{EnhanceNet} & 28.56 & 25.77 & 24.93   \\
Bicubic        & 28.42 & 26.1  & 25.96   \\ \bottomrule
\end{tabular}
\end{table}

\section{Problem Statement}
Our goal is to increase the resolution of retinal fundus images.
We discuss two similar problems here:
\begin{itemize}
\item Increasing the spatial resolution of images.
  We consider images with a low number of pixels and increase the number of pixels by a factor of 16.
  In the literature, this problem is called single image super resolution.
\item Denoising images.
  For this case we try to increase the quality of images by removing blur.
\end{itemize}


\section{Architecture}
We use a modified \textbf{LapSRN} architecture which was presented in~\cite{LapSRN}.
The network utilises a Laplacian pyramid, it first performs 2x upscaling and then 4x upscaling.

This architecture consists of two branches.
The first branch is called \textbf{image reconstruction branch} and upscales the low resolution image using simple upscaling filters.
The second branch extracts features from the low resolution image using stacked convolutions and predicts the \textbf{residual} for each upscaled image of the first branch.

Our model differs in two ways from the original architecture:
\begin{itemize}
\item We use three-channel \textsc{rgb} images both as input and output instead of working only on the Y-channel.
\item Instead of transposed convolutions we use resize-convolution blocks.
  They are composed of a nearest neighbour interpolation, a reflect-pad of size one and a standard convolution filter.
  Empirical evidence suggest that they lead to fewer interpolation artefacts than transposed convolution, especially when they are used in a \textsc{gan} framework\cite{deconvolution}.
\item We only consider 4x upscaling here.
  The approach presented here can be easily extended to larger upscaling ratios.
\end{itemize}
The resulting architecture can be seen in figure xx.
All convolutions in the generator have 64 $3 \times 3$ filters, a stride of one and a padding of one.

For the deblurring case, the upsampling block is removed from the feature extraction pipeline.
Additionally, the image reconstruction stage is replaced by the blurred high-resolution images.
We thus use a similar architecture and keep the residual learning and the stepwise(?) reconstruction.

We use the PatchGan discriminator presented in~\cite{PatchGAN}.
This is a fully convolutional discriminator that penalizes on a basis $70\times70$ patches.
In our case this can be interpreted as a very coarse texture loss.


\section{Loss functions}
In this work we combine three loss functions:
A saliency weighted $L_1$-loss ensures the faithful reconstruction of important areas, a perceptual loss optimises visual similarity and an adversarial loss reconstructs high-frequency image components.
This ensemble of loss functions is inspired by~\cite{SaliencyGAN}

While the images in standard super-resolution tasks do not share a lot of common structure, retinal fundus images are visually similar to each other.
This can be used to create hand-crafted saliency maps, that highlight relevant pixels.
We use maps that are similar to the ones of~\cite{SaliencyGAN}.
These saliency map consists of two components: the curvature and local entropy.

The isophote curvature highlights fine structure in the image and can be computed as
\begin{equation}
 I_c = \frac{\img_{xx} \img_y^2 + f_{yy} \img_x^2 - 2 \img_{x} \img_{xy} \img_{y} }{(\img_x^2 + \img_y^2)^{1.5}},
\end{equation}
where \(\img\) is the intensity of the image and its subscript represents its derivatives.
A derivation and discussion of this curvature function can be found in~\cite{Curvature}.
We approximate the image derivatives by convolving the image with the derivative of Gaussians with \(\sigma = 1\).

The local entropy highlights pixels that are in a neighborhood that contain more information.
We compute the entropy for each pixel by
\begin{equation}
  \label{eq:entr}
  I_e = - \sum_{s \in P_s} p(s_i) \log(p(s_i)),
\end{equation}
where \(p\) is the probability that a pixel \(s\) has an intensity of \(s_i\) in a patch \(P_s\).
The probability $p(s_i)$ is estimated by a normalised histogram with eight bins of equal size, computed for each patch seperately.
A patch is composed of the pixel itself and its surrounding neighborhood of size \(7 \times 7\).
We then convolve the image with a Gaussian filter with \(\sigma = 0.5\) to remove high-frequency noise.
Finally we normaliuue $I_e$ such that it is in the range \([0, 1]\) and use $1 - I_e$ for our saliency map.
This highlights compact regions of the image.

We then compute the uniqueness of each pixel of a feature map \(f\) with the function
\begin{equation}
  \label{eq:uniq}
  U(f) = \sum_{o \in P_c} w(c, o) \vert f(c) - f(o) \vert,
\end{equation}
where \(c\) corresponds to the pixel in the middle of a patch \(P_c\) of size \(7 \times 7\).
The function \(w(a,b) = \exp(d(a, b))\) weights the pixels by their Euclidean distance \(d\).
Pixels that are farther away from the center have a smaller influence on the uniqueness.
All uniqueness maps are normalized to the range \([0,1]\).

Finally, we combine the uniqueness maps of both components linearly
\begin{equation}
  \label{eq:saliency}
  I_{\text{sal}} = 0.4 \cdot U(I_c) + 0.6 \cdot U(1 - I_e)
\end{equation}
to obtain the saliency map for each image.

We then use this map to weigh the pixel-wise error.
We propose a weighted version of the Charbonnier loss\todo{Cite charbonnier loss}
\todo{define weighted l1 loss}
\begin{align}
\label{eq:charbonnier}
    L_{\text{sal}}( \hat{\bm{\img}}, \bm{\img}) = \sum_n p \left( \bm{\hat{\img}} - \bm{\img} \right),
  \intertext{with }
  \label{eq:charbonnier-p}
    p(\bm{x}) = I_{\text{sal}} \cdot \sqrt{ \bm{x}^2 + \varepsilon},
\end{align}
where $\varepsilon$ is a parameter set empirically to $1e-6$ for our experiments and all mathematical operations in \cref{eq:charbonnier-p} are element-wise.
It is a derivable\todo{correct word here?} approximation of the $L_1$ loss which leads to sharper edges compared to a standard \textsc{mse}-loss~\cite{LapSRN}.

Secondly, we use a perceptual loss.
The idea is to not compare images pixel-wise but consider difference of filtered images.
This loss can be defined for an arbitrary filter map as
\begin{equation}
  \label{eq:perceptual-loss}
  L_p = \Vert \phi( \bm{\hat{y}} ) - \phi (\bm{y}) \Vert_2^2 .
\end{equation}
We use filter maps $\phi$ taken from the first and second pooling layer of a pre-trained VGG-16~\cite{Vgg} network.
This leads to visually better images but decreased PSNR.
See~\cite{PerceptualLoss} for motivation and details.

We use an adversarial loss as our final building block.
Training the model now resembles a two-player game between a generator an a discriminator network.
The discriminator tries to decide, whether an image is a ground-truth high-resolution image or an low resolution image that was upsampled by the generator.
The generator is trained to fool the discriminator.
This is realized as the optimization problem~\cite{GAN}
\begin{align}
 \min_G \max_D \mathbb{E}_{x \sim P_r} \left[ \log (D({x})) \right] +
  \mathbb{E}_{\hat{\bm{x}} \sim P_g} \left[  \log (1 - D(\hat{\bm{x}})) \right].
\end{align}
To improve stability, the generator minimizes
\( - \mathbb{E}_{\hat{\bm{x}} \sim P_g} \left[ \log (D(\hat{\bm{x}})) \right]\)
instead.
The patch-discriminator does not predict a single probablity per image but rather one per image patch.
Thus, the probability distributions $P_r$ and $P_g$ correspond to the distribution of real high-resolution patches and super-resolved patches respectively.
This discriminator design is able to discover high-frequency image details while relying on the other two losses for low-frequency content~\cite{PatchGAN}.

The first implementation that used an adversarial loss for single-image super-resolution was~\cite{SRGAN} which resulted in visually pleasing images.
\todo{Describe imagined artefacts?}
\todo{Only standard GAN? or also wgan?}

The total loss function is then
\begin{equation}
  \label{eq:total-loss}
\sum_{\img, \hat{\img}}
\frac{1}{3 W_{\img} H_{\img}}
  \left( 5 L_{\text{sal}} (\img, \hat{img}) + 0.12  L_p(\img, \hat{\img}) \right) + 0.01 L_a(\img, \hat{\img}),
\end{equation}
where the sum runs over all model outputs \(\hat{\img}\) and the corresponding ground truth \(\img\).
All losses are normalised by number of channels (3), image height \(H_{\img}\) and width \(W_{\img}\).
The weights for saliency and adversarial loss are chosen such that they are of roughly the same order, the weight for the adversarial loss is chosen empirically.

\section{Implementation \textit{\&} Training}
\todo{Weight init.}
We use the Messidor dataset~\cite{Messidor} which consists of 1200 high resolution fundus images.
The black borders from the images were removed.
We used 80\% of the dataset for training, the rest was used for validation.
The high resolution images are augmented by
\begin{itemize}
\item A random scaling with a factor uniformly distributed between $0.5$ and $1.0$.
  We rescale using bicubic downsampling.
\item Random crop of size $128 \times 128$ chosen by rejection sampling:
If a crop contains more than $50\%$ black pixels (gray-scale value smaller than 90) or no vessels it is rejected and another crop is chosen.
The vessels are detected using a pre-computed Frangi filter~\cite{Frangi}.
We consider a crop to contain no vessels when less than $64$ pixels are marked as vessels by the Frangi filter.
This threshold was chosen empirically.
\item Random rotation by either $0, 90, 180, \text{or } 270$ degrees.
  These roations do not change the size of the image.
\item Random vertical and horicontal flip, each with a probability of $50\%$.
\item Specular reflections with a probability of $25\%$.
This is simulated by increasing the intensity of the image in a circular mask (post-processed with a Gaussian filter) by a random intensity.
The specular reflection mirror the usage of light sources during operations.
\end{itemize}
Note that all augmentations (except the specular reflections) are also applied to the pre-computed saliency-maps and vessel segmentations.

We obtain low resolution images are obtained using
\begin{description}
\item[Super-resolutuon] Gaussian blur (with a maximum radius of two) followed by bicubic downsampling or
\item[Denoising] Gaussian blur (with a maximum radius of three).
\end{description}
The blurring is done by first selecting a random total blur radius, sampled uniformally between zero and the maximum blur.
We then distribute the blur such that the intermediate image is blurred with half strength.
This intermediate blur can be computed by \( \left( \text{radius}_{\text{total}} / \sqrt{2} \right)\), using the fact that a convolution of two Gaussians is a Gaussian.
Finally, images and saliency maps) are converted to tensors of range 0 to 1.
Images are scaled by subtracting
\(
\begin{bmatrix}
 0.485 & 0.456 & 0.406 
\end{bmatrix}
\)
and dividing by
\(
\begin{bmatrix}
0.229 & 0.224& 0.225
\end{bmatrix}
\).
This is the normalisation that is expected from the pre-trained VGG-16 network.

We use a batch size of $64$ for all experiments.
An epoch thus consists of $15$ gradient updates.

Both networks are optimized by \textsc{adam}~\cite{Adam} with an initial learning rate of $1r-3$.
We use a weight decay of $1e-4$ for the generator.
The generator is trained for $9999$ epochs, the learning rate is divided by ten every $3333$ epochs.
We first train solely the generator without the adversarial loss.
The resulting network is then used to initialise the adversarial training, which continues for another $9999$ epochs.
We validate after every $10$ epochs.
\todo{Mention LR-Schedule for deblurring!}

\section{Evaluation}
The evaluation of image restoration networks is difficult, as commonly used metric do not correlate with perceptual quality.
We use the \textsc{mse}-based metric \textsc{psnr}, defined by
\begin{equation}
  \label{eq:psnr}
  \cdots,
\end{equation}
to verify that individual pixels are faithfully restored.
Additionally we use the metric \textsc{ssim}~\cite{Ssim}, which compares images with sliding windows (is that true?).

For our chosen application, the correct reconstruction of image gradients is important, for example because we want to achieve a clear border between areas where the membrane still exists and other areas.
To do this, we compute the gradient magnitude with a Sobel filter and compare the reconstruction.

Similarly to~\cite{SaliencyGAN} we also use vessel segmentation as a proxy for perceptual quality.
For this we use both a Frangi filter~\cite{Frangi}, which is a simple method that is not robust, and the retina-unet~\cite{RetinaUnet}.
We evaluate this on the testing set of the DRIVE dataset~\cite{Drive}.

Finally, we look at the visual quality of the upscaling of intraoperative pictures.
These images differ strongly to the images seen by the network, it is thus an instance of transfer learning.

We compare the super resolution network with bicubic interpolation.
The deblurring network is compared to a standard U-Net architecture, trained with the same parameters as our network with \textsc{mse} loss.
In the implementation that we are using, the feature maps for the skip connections are not cropped but rather padded such that they have the same shape.
This is a common architecure for bio-medical applications and similar architectures have been used successfully as generators for image-translation tasks~\cite{PatchGAN}.
The UNet was trained with the same optimizer configuration and an identical learing rate schedule.

\begin{table}[]
\centering
\caption{Results for models on Drive dataset.}

\label{tab:results-sr}
\begin{tabular}{@{}lllll@{}}
\toprule
Model          & PSNR  & SSIM & Sobel-\textsc{mse} $\times 10^4$ & \textsc{auc} \\ \midrule
Ground Truth & $\infty$ & 0.0 & 0.0 & 0.0 \\
Bicubic & $0.0 & 0.0 & 0.0 & 0.0 \\
Full & 0.0 & 0.0 & 0.0 & 0.0 \\
Saliency + Perceptual & 0.0 & 0.0 & 0.0 & 0.0 \\
Saliency & 0.0 & 0.0 & 0.0 & 0.0 \\
Perceptual & 0.0 & 0.0 & 0.0 & 0.0 \\
\bottomrule
\end{tabular}
\end{table}

\todo{PSNR table}
\todo{Segmentation table, Frangi AND Retina-U-NEt}
\todo{Intraoperative evaluation}
\section{Summary}
summary here

\printbibliography
\end{document}