\documentclass{scrartcl}
\usepackage[utf8]{inputenc}
\usepackage[T1]{fontenc}

\usepackage[american]{babel}
\usepackage[autostyle, english = american]{csquotes}
 \usepackage[%
   backend=biber]{biblatex}
 \addbibresource{../bibliography.bib}
\usepackage[final]{microtype}
\usepackage{todonotes}
\usepackage{caption}
\usepackage{subcaption}
\usepackage{bm}
\usepackage{amsmath}
\usepackage{mathtools} % for \mathclap
\usepackage{hyperref}
\usepackage[noabbrev]{cleveref}
\newcommand{\creflastconjunction}{, and\nobreakspace} % use Oxford comma
%\usepackage{placeins} % for floatbarrier

\usepackage{tikz}
\usetikzlibrary{arrows, positioning, shapes.geometric}
\usetikzlibrary{calc}

\begin{document}
\title{Deep-Learning-based Image Denoising in Ophthalmology}
\author{Lukas Krenz}

\maketitle

\section{Modeling Considerations}

\subsection{Architectures}
\label{sec:architecturs}
Differentiate between methods optimised for quality and speed.
Quality mostly feeds in a upsampled image, filters for in HR-space.
Efficiency uses low-resolution image as input, upscales at end.

Otherwise:
Residual blocks, pyramid structure~\cite{LapSRN}, residual learning.

\subsection{Loss functions}
\label{sec:loss}

\begin{description}
\item[MSE] Standard pixelwise mse, optimizes PSNR, leads to blurry images
\item[Charbonnier-Loss] Pixelwise (smoothed) $L_1$ error, leads to better edges, e.g.~\cite{LapSRN}
\item[Saliency Loss] Weigh pixels by their importance, hand tuned weigts.
  E.g. vessel curvature
\item[Perceptive Loss] Don't compare pixel-wise but consider learned filters from a different network.
  Mostly VGG-19. Leads to visually better images but decreased PSNR
\item[Adversarial Loss] Optimize linear combination of other loss and loss function learned by d/critic network.
  Leads to visually pleasing images, reduced PSNR and might lead to ``imagined'' artifacts.
  Maybe dubious in a medical setting.
\end{description}
\subsection{Evaluation}
Either use metrics such as PSNR/SSIM, visually, or use efficiency as pre-processing, e.g. improved segmentation results
\label{sec:evaluation}

\subsection{Relevant results}
Difficult to compare, visual quality does not correspond directly to improved error!

\printbibliography
\end{document}